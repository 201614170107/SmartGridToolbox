\documentclass[11pt]{article}

\usepackage{amsmath}
\usepackage{mathtools}
\usepackage{geometry}
\geometry{a4paper}
\usepackage[parfill]{parskip}
\usepackage{graphicx}
\usepackage{amssymb}
\usepackage{epstopdf}
\usepackage{listings}
\lstset{language = C++}
\usepackage{url}

\newcommand{\re}[1]{{{#1}_R}}
\newcommand{\im}[1]{{{#1}_I}}

\title{Power Flow Theory}
\author{Dan Gordon}
\date{}

\begin{document}
\maketitle

\section{Nomenclature}
\begin{align*}
y_{ik} &= g_{ik} + jb_{ik} = \text{complex admittance along branch $ik$.} \\
Y_{ik} &= G_{ik} + jB_{ik} = \text{nodal admittance matrix element $i, k$.} \\
&= 
	\begin{cases}
		-y_{ik}&\text{if $i \ne k$} \\
		y_i + \sum_l y_{il}& \text{if $i = k$}
	\end{cases} \\
V_i &= \text{complex voltage at bus $i$.} \\
M_i &= \text{voltage magnitude at bus $i$, $M_i := |V_i|$.} \\
I_{\text{br}i} &= \text{complex current injection from branches at bus $i$.} \\
I_{\text{ld}i} &= \text{total complex current injection from load at bus $i$.} \\
I_{\text{bus}i} &= \text{complex current injection due to the bus at bus $i$.} \\
S_{\text{bus}i} &= \text{complex power injection due to bus $i$.} \\
S_{ci} &= \text{constant power component of load at bus $i$.} \\
I_{ci} &= \text{constant current injection component of load.} \\
y_{ci} &= \text{constant impedance component of load.} \\
\delta_{ik} &= \text{the Kronecker delta, $\delta_{ik} = 1$ if $i = k$, 0 otherwise.}
\end{align*}

\section{Power flow equations in current form}
For simplicity, we write all bus quantities as injections \emph{into} the bus. Thus a normal load will use quantities expressed as negative injections, while a generator will have positive injections.

Each bus has an associated load, containing a constant power component, a constant current component, and a constant shunt impedance current. As well as receiving injections from branches and loads, busses provide their own injections according to their bus type. PQ busses provide a specified complex power injection. PV busses instead keep the voltage magnitude of the bus constant while providing a specified real power injection. Slack busses keep the complex voltage of the bus constant.

The total current injection into bus $i$ is:
\begin{align}
I_i &= I_{\text{br}, i} + I_{\text{ld}i} + I_{\text{bus}, i} = 0\\
I_{\text{bri}} &= -\sum_{k=0}^NY_{ik}V_k \\
I_{\text{ld}i} &= \frac{S^*_{ci}}{V^*_i} + I_{ci} - y_{ci}V_i \\
I_{\text{bus}i} &= \frac{S^*_{\text{bus}i}}{V^*_i}
\end{align}
which is zero, due to Kirchoff's current conservation law. Thus,
\begin{align}
I_i &= \frac{S^*_{ci} + S^*_{\text{bus}i}}{V^*_i} + I_{ci} - y_{ci}V_i - \sum_{k=0}^NY_{ik}V_k = 0
\end{align}
or, absorbing $y_c$ into $Y$ and $S_{\text{bus}}$ into $S_c$ we have
\begin{align}
I_i &= \frac{S'^*_{ci}}{V^*_i} + I_{ci} - \sum_{k=0}^NY'_{ik}V_k = 0
\end{align}
where
\begin{align}
	Y'_{ik} &= Y_{ik} + y_{ci}\delta_{ki}
\end{align}
and $S'_c = S_c + S_{\text{bus}}$.

Real and imaginary components are:
\begin{align}
\re{I}_i &= \frac{P'_{ci}\re{V}_{i} + Q'_{ci}\im{V}_{i}}{M^2_i} + \re{I_c}_{i} + \sum_{k=0}^N\left(-G'_{ik}\re{V}_k + B'_{ik}\im{V}_k\right) \\
\im{I}_i &= \frac{P'_{ci}\im{V}_{i} - Q'_{ci}\re{V}_{i}}{M^2_i} + \im{I_c}_{i} + \sum_{k=0}^N\left(-G'_{ik}\im{V}_k - B'_{ik}\re{V}_k\right) = 0
\end{align}

For PQ busses, the unknowns are the real and imaginary parts of $V$, so this equation can be solved using the Newton-Raphson method. Letting the function to which we want to find the zero be $f = \{\re{I}, \im{I}\}$, the unknows be $x = \{\re{V}, \im{V}\}$, we wish to solve $f(x) = 0$. Using the Jacobian
\begin{align}
J_{ik}(x) = \frac{\partial f_i(x)}{\partial x_k}
\end{align}
the NR method calculates the update to $x$ at each iteration as the solution to the linear equations
\begin{align}
-f_{(n)} &= J(x_{(n)})(x_{(n+1)}-x_{(n)}) = J(x_{(n)})\Delta x_{(n,n+1)}
\end{align}
The Jacobian is given by:
\begin{align}
\frac{\partial \re{I}_i}{\partial \re{V}_{k}} &= \left[-\frac{2\re{V}_i(P'_{ci}\re{V}_i + Q'_{ci}\im{V}_i)}{M_i^4} + \frac{P'_{ci}}{M_i^2} \right]\delta_{ik} - G'_{ik} \\
\frac{\partial \re{I}_i}{\partial \im{V}_{k}} &= \left[-\frac{2\im{V}_i(P'_{ci}\re{V}_i + Q'_{ci}\im{V}_i)}{M_i^4} + \frac{Q'_{ci}}{M_i^2} \right]\delta_{ik}  + B'_{ik} \\
\frac{\partial \im{I}_i}{\partial \re{V}_{k}} &= \left[-\frac{2\re{V}_i(P'_{ci}\im{V}_i - Q'_{ci}\re{V}_i)}{M_i^4} - \frac{Q'_{ci}}{M_i^2} \right]\delta_{ik}  - B'_{ik} \\
\frac{\partial \im{I}_i}{\partial \im{V}_{k}} &= \left[-\frac{2\im{V}_i(P'_{ci}\im{V}_i - Q'_{ci}\re{V}_i)}{M_i^4} + \frac{P'_{ci}}{M_i^2} \right] \delta_{ik} - G'_{ik} \\
\end{align}

For a PV bus $k$, the power flow equations also hold, but with $Q_{\text{bus}}$ being considered as a variable rather than a constant, and an extra constraint:
\begin{align}
\Delta M^2_k = \re{V}_{k}^2 + \im{V}_k^2 - M^2_{\text{PVk}} = 0
\end{align}

If we begin with this constraint being satisfied, and enforce during NR that
\begin{align}
\re{\Delta V}_k &= -\frac{\im{V}_k\im{\Delta V}_k}{\re{V}_k}
\end{align}
then the constraint will always continue be satisfied. Thus, $\re{\Delta V}$ may be eliminated from the NR equations. (This result may easily be more formally derived by considering the Jacobian terms in the constraint). Any terms that would normally be in the $\re{\Delta V}_k$ column of the Jacobian will then need to be multiplied by $-\im{V}_k/\re{V}_k$ and then shifted to the $\im{\Delta V}_k$ column.

The $\re{\Delta V}_k$ variable will need to be replaced with a column for the extra variable $\Delta Q_{\text{bus}k}$.  The extra terms are:
\begin{align}
\frac{\partial \re{I}_i}{\partial Q_{\text{bus}k}} &= \frac{\im{V}_k}{M_k^2}\delta_{ik} \\
\frac{\partial \im{I}_i}{\partial Q_{\text{bus}k}} &= -\frac{\re{V}_k}{M_k^2}\delta_{ik}
\end{align}

For PV busses, $I$ is the same as that of $PQ$ busses.

\end{document}